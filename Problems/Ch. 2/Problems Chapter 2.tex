\documentclass[a4paper]{article}
% Typically the 'article' class is appropriate for assignments.
% And we print it on a4, so we include that as well.

\usepackage{a4wide}
% To decrease the margins and allow more text on a page.

\usepackage{graphicx}
\graphicspath{./Images/}
% To deal with including pictures.

\usepackage{color}
% To add color.

\usepackage{amsmath}
% For displaying math blocks.

\usepackage{enumerate}
% To provide a little bit more functionality than with LaTeX's default
% enumerate environment.

\usepackage{array} 
% To provide a little bit more functionality than with LaTeX's default
% array environment.

\usepackage[american]{babel}
% Use this if you want to write the document in US English. It takes care of
% (usually) proper hyphenation.
% If you want to write your answers in Dutch, please replace 'american'
% by 'dutch'.
% Note that after a change it may be that the first compilation of LaTeX
% fails. That is normal and caused by the fact that in auxiliary files
% from previous runs, there may still be a \selectlanguage{american}
% around, which is invalid if 'american' is not incorporated with babel.

\usepackage{tikz}
\usetikzlibrary{bayesnet}
% For drawing Bayesian networks.

\usepackage{listings}
% For including code snippets with proper syntax highlighting and formatting.


\title{Bayesian Networks and Causal Inference \\ Lecture Notes Problem Answers \\ Chapter 2}
\author{Daan Brugmans \\ S1080742}
\date{\today}

\begin{document}
\maketitle
\section*{Problem 2.1}
\begin{enumerate}
    \item \begin{align*}
        P(A = 0) &= \sum_b P(A = 0, b) \\
        &= P(A = 0, B = 0) + P(A = 0, B = 1) \\
        &= 0.3 + 0.4 \\
        &= 0.7 \\
        \\
        P(B = 0) &= \sum_a P(a, B = 0) \\
        &= P(A = 0, B = 0) + P(A = 1, B = 0) \\
        &= 0.3 + 0.2 \\
        &= 0.5
    \end{align*}
    \item \begin{align*}
        P(A = 0 | B = 0) &= \frac{P(B = 0 | A = 0)P(A = 0)}{P(B = 0)} \\
        &= \frac{P(B = 0, A = 0)}{P(A = 0)} \\
        &= \frac{0.3}{0.7} \\
        &\approx 0.429
    \end{align*}
    \item \begin{align*}
        P(a | B = 0)
    \end{align*}
    \item \begin{align*}
        P(B = 0 | A = 0)
    \end{align*}
\end{enumerate}

\end{document}
